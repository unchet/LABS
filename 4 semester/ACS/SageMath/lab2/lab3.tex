\section{Приведение уравнения поверхности второго рода к каноническому виду}

\begin{sagesilent}
    plot_range = 10.0
    u(x, y, z) = 11 * x ^ 2 - 2 * x * y - 2 * x * z + 2 * y * z + 9 * z ^ 2 - 4 * x + y + z
\end{sagesilent}

Исходное уравнение поверхности второго рода:

$\sage{u(x, y, z)} = 0$

\begin{sagesilent}
    # Из уравнения поверхности получим коэффициенты
    a11 = 11
    a22 = 0
    a33 = 9

    a12 = -2 / 2
    a13 = -2 / 2
    a23 = 2 / 2

    a0 = 0
    a1 = -4 / 2
    a2 = 1 / 2
    a3 = 1 / 2

    matrA = matrix([
    [a11, a12, a13],
    [a12, a22, a23],
    [a13, a23, a33]
    ])

    vecA = vector([a1, a2, a3])
\end{sagesilent}

Матричный вид уравнения:

$\sage{LatexExpr("A = " + latex(matrA))}$

И вектор линейных коэффициентов:

$\sage{LatexExpr("a = " + latex(vecA))}$

\begin{sagesilent}
    # Характеристический многочлен
    var("l")
    matrAm = matrA - l * identity_matrix(3)
    detA = det(matrAm).simplify_full()
\end{sagesilent}

Составим характеристический многочлен:

$\sage{LatexExpr(r'\det(A - l * E) = ' + latex(detA))}$

\begin{sagesilent}
    solutions = solve(detA == 0, l)
    eigenvalues = []
    for i in range(len(solutions)):
        eigenvalues.append(solutions[i].rhs().real().n(digits = 8))
\end{sagesilent}

Корни характеристического уравнения:

$\sage{LatexExpr("l_0 = " + str(eigenvalues[0]) + r', l_1 = ' + str(eigenvalues[1]) + r', l_2 = ' + str(eigenvalues[2]))}$

\begin{sagesilent}
    eigenvectors = []
    var("x1 x2 x3")
    for i in range(len(eigenvalues)):
        matrB = matrA - eigenvalues[i] * identity_matrix(3)
        system = []
        for j in range(3):
            system.append(matrB[j][0] * x1 + matrB[j][1] * x2 + matrB[j][2] * x3 == 0)
        
        system[1] = system[1] - system[0] * matrB[1][0] / matrB[0][0]
        system[2] = system[2] - system[0] * matrB[2][0] / matrB[0][0]
        system[2] = system[2] - system[1]  * (system[2].lhs() / system[1].lhs())
       
        if(system[2].lhs() == 0 and system[2].rhs() == 0):
            system[2] = (x3 == 1)
        else:
            print("Ранг матрицы равен трём. Что-то пошло не так")

        eigenvec = vector([k.rhs().n(digits = 10) for k in solve(system, x1, x2, x3)[0]])
        ans = matrB * eigenvec
        eigenvectors.append(eigenvec.n(digits = 8))
\end{sagesilent}

Получим собственные векторы:

$\sage{LatexExpr(r'\lambda_0 = ' + str(eigenvalues[0].n(digits = 8)) + ', s_0 = ' + str(eigenvectors[0].n(digits = 8)))}$

$\sage{LatexExpr(r'\lambda_1 = ' + str(eigenvalues[1].n(digits = 8)) + ', s_1 = ' + str(eigenvectors[1].n(digits = 8)))}$

$\sage{LatexExpr(r'\lambda_2 = ' + str(eigenvalues[2].n(digits = 8)) + ', s_2 = ' + str(eigenvectors[2].n(digits = 8)))}$

\begin{sagesilent}
    matrS = matrix(eigenvectors)
\end{sagesilent}

Матрица перехода:

$\sage{LatexExpr("S = " + latex(matrS.n(digits = 8)))}$

\begin{sagesilent}
    for i in range(len(eigenvectors)):
        eigenvectors[i] = eigenvectors[i] / sqrt(eigenvectors[i].dot_product(eigenvectors[i]))
    matrSn = matrix(eigenvectors) 
\end{sagesilent}

нормированная матрица перехода:

$\sage{LatexExpr("S^* = " + latex(matrSn.n(digits = 8)))}$

\begin{sagesilent}
    matrDiag = matrix([
    [eigenvalues[0], 0, 0],
    [0, eigenvalues[1], 0],
    [0, 0, eigenvalues[2]]
])
    matrCheck = matrSn.T * matrDiag * matrSn
\end{sagesilent}

Составим диагональную матрицу для матрицы А

$\sage{LatexExpr("Diag = " + latex(matrDiag.n(digits = 8)))}$

$\sage{LatexExpr("Sn^T * Diag * Sn = " + latex(matrCheck.n(digits = 8)))}$

Видим, что эта матрица соответствует изначальной, значит все вычисления были верными

\begin{sagesilent}
    vecA1 = matrSn.T * vecA
\end{sagesilent}

Новый вектор линейных коэффициентов:

$\sage{vecA1}$

\begin{sagesilent}
    v(x, y, z) = 0
    variables = [x, y, z]
    for i in range(len(variables)):
        v += matrDiag[i][i] * variables[i] ^ 2 + variables[i] * vecA1[i]
\end{sagesilent}

После приведения к каноническому виду получим:

$\sage{v(x, y, z) == 0}$

\begin{sagesilent}
    coef = []
    for i in range(len(vecA1)):
        coef.append((vecA1[i] / (2 * matrDiag[i][i])) ** 2)
        if (matrDiag[i][i] * vecA1[i] < 0):
            coef[i] = -coef[i]

    v(x, y, z) = 0
    for i in range(len(variables)):
        v += matrDiag[i][i] * (variables[i] + coef[i]) ^ 2 - matrDiag[i][i] * abs(coef[i])
\end{sagesilent}

После замены переменных получим:

$\sage{v(x, y, z) == 0}$

Сравним графики исходного уравнения поверхности второго рода и канонического уравнения после замены переменных.

График функции $\sage{u(x, y, z)} = 0$
\begin{center}
  \sageplot[trim=50 50 50 50, clip, width=10cm][png]{implicit_plot3d(u, (-plot_range, plot_range), (-plot_range, plot_range), (-plot_range, plot_range), color="orange")}
\end{center}

График функции $\sage{v(x, y, z)} = 0$
\begin{center}
  \sageplot[trim=50 50 50 50, clip, width=10cm][png]{implicit_plot3d(v, (-plot_range, plot_range), (-plot_range, plot_range), (-plot_range, plot_range), color="orange")}
\end{center}